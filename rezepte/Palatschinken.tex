\chapter*{Palatschinken}
\begin{multicols}{2}
 {\Large Zutaten}
 \begin{Zutaten}
		\item \sfrac{1}{2}l Joghurtbecher Mehl
		\item 1Prise Salz
		\item \sfrac{1}{2}l Milch
		\item 4 Eier
		\item \sfrac{1}{8}l Bier (optional statt selber Menge Milch)
		
		
		
\end{Zutaten}
	
\columnbreak
\addPicture{palatschinken.jpg}
\end{multicols}

{\Large Zubereitung} \newline
\begin{addmargin}[1cm]{0cm}
	Einen \sfrac{1}{2}l Joghurtbecher mit Mehl füllen, eine Prise Salz dazu und in einen Mixer geben.\newline
	Bevor man das Gerät einschaltet müssen die anderen Zutaten vorbereitet werden. Wenn durch das späte
	Zubereiten das zuführen der Zutaten verzögert wird bekommt man "Boban".\newline
	Also sollte man nun zwei Gefäße mit \sfrac{1}{4} Milch, ein Gefäß mit den 4 aufgeschlagenen Eiern und
	optional ein Gefäß mit \sfrac{1}{8}l Bier (dann jedoch weniger Milch!) haben.\newline
	Nun muss zuerst wärend der Mixer läuft \sfrac{1}{4}l Milch zugeführt werden. Danach kommen gleich die Eier
	dazu. Alles gut durchmischen.\newline
	Nun kurz eine Pause machen und mit einer Teigkarte die Ränder herunter putzen.
	Danach die restlichen Zutaten dazu und nochmal gut Mixen.\newline
	Alles gut in eine Schüssel putzen und nochmal die konsistenz prüfen. Falls zu dickflüssig kann noch
	Milch ergänzt werden.\newline
	Nun in einer Pfanne Palatschinken machen.\newline\newline
	
	Die Pfanne kann mit Öl, Butter oder Butterschmalz eingefettet werden. Den besten Geschmack erzielt
	man mit Butter, jedoch stinkt diese beim heraus braten der Palatschinken.\newline
	Tipp: Zum einfetten eine Küchenrolle in Fett tränken und Pfanne ein schmieren.
	
	
	
\end{addmargin}
