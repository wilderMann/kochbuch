\chapter*{Kokos-Ingwer-Karottencremesuppe}
\begin{multicols}{2}
 {\Large Zutaten}
 \begin{Zutaten}
		\item 1kg Karotte, gewaschen, in dünne Räder
		geschnitten
		\item 1 Zwiebeln, fein gehackt
		\item 1 Stück Ingwer, gut daumengroß oder größer
		\item 4 mittel-große Süßkartoffeln,
		normale Kartoffeln tuns auch
		\item 1 Dose Kokosmilch
		\item \sfac{1}{2}l Gemüsebrühe
		\item 2 EL Limettensaft
		\item 1 EL Honig
		\item Salz und Pfeffer
		\item 1 Bund Koriandergrün
		
		
		
				
		
\end{Zutaten}
\columnbreak
\addPicture{kokosIngwerKarottencremesuppe.jpg}
\end{multicols}

{\Large Zubereitung} \newline
\begin{addmargin}[1cm]{0cm}
	Ergibt etwa 5 Teller Suppe.\newline\newline
	
	Zwiebeln in etwas Öl bei mittlerer Hitze 5min dämpfen.\newline Karotten und Ingwer beigeben, weitere
	5min dämpfen. Dann die Gemüsebrühe und die geschälten, gewürfelten Kartoffeln hinzugeben und
	kochen, bis das Gemüse weich ist \left(ca. 25 min\right).\newline Abkühlen lassen. Die Gemüsesuppe mit dem Pürierstab pürieren.\newline Anschließend Kokosmilch, Limettensaft und Honig hinzugeben und nur noch leicht erwärmen.\newline
	Mit Salz und Pfeffer abschmecken.\newline Vor dem Servieren mit gehackten Korianderblättern bestreuen.\newline
	Falls die Suppe zu dick ist kann mit Gemüsebrühe nachverdünnt werden.
	
	
	
	
	
\end{addmargin}
