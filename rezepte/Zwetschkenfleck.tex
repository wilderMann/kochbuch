\chapter*{Zwetschkenfleck mit Streusel}
\begin{multicols}{2}
 {\Large Zutaten}
 \begin{Zutaten}
		\item 25dag Mehl
		\item 12dag Butter
		\item 4dag Zucker
		\item 1P. Germ
		\item \sfrac{1}{16}l Milch
		\item 2 Dotter
		\item Zwetschken zum belegen
		\item 10dag glattes Mehl
		\item 10dag Zucker
		\item 10dag geriebene Nüsse
		\item 12dag Butter
		
		
\end{Zutaten}
	
\columnbreak
\addPicture{zwetschkenfleck.jpg}
\end{multicols}

{\Large Zubereitung} \newline
\begin{addmargin}[1cm]{0cm}
	Dampferl: 1P. Germ mit 4dag Zucker Mischen, verflüssigt sich dann. Wenn flüssig dann Milch dazu.\newline\newline
	Mehl, Butter und Prise Salz abbröseln (miteinander so lange rühren bis sich aus einem großen Klumpen
	kleine Brösel gebildet haben).\newline
	Dampferl und Dotter dazu und durchmischen.
	Den Teig dann ca. 1h im Kühlschrank ruhen lassen.\newline
	Danach auf befettetem Blech ausrollen und dicht mit entkernten geviertelten Zwetschken belegen.\newline
	Danach mit Streusel bestreuen und 35-40min bei 150°C ins kalte Rohr bei unter- und Oberhitze.\newline
	Hellbraun backen. Besser kürzer als zu lange da sonst trocken.
	Für Streusel, letzten 4 Zutaten einfach abbröseln.\newline
	
\end{addmargin}
