\chapter*{(Lachs)Risotto}
\begin{multicols}{2}
 {\Large Zutaten}
 \begin{Zutaten}
		\item Olivenöl
		\item 1 Stk Butter
		\item 2 Becher Risotto Reis (für 4 Personen)
		\item 1 Lachsfilet
		\item Zwiebel
		\item Lauch
		\item Brokkoli
		\item 1l Hühner Brühsuppe (1 Würfel)
		\item Weißwein
		\item Zitrone
		\item (Zitronen)Pfeffer
		\item Parmesan
		
		
		
		
		
\end{Zutaten}
\columnbreak
%\addPicture{shahiPaneer.jpg}
\end{multicols}

{\Large Zubereitung} \newline
\begin{addmargin}[1cm]{0cm}
	Lachsfilet mit Zitronensaft übergießen, mit Pfeffer (bestenfalls Zitronenpfeffer) würzen und eine
	Seite des Lachs mit Mehl stauben. \newline
	Butter in einer Pfanne erhitzen und den Lachs nicht zu scharf anbraten und zerteilen.
	Danach geschlossen auf der heißen Platte stehen lassen! \newline
	Nun wenden wir uns dem Risotto zu. Im Wasserkocher 1l Wasser zum Kochen bringen, in einem Topf auf
	dem Herd heiß halten und einen Hühner Brühwürfel dazu. \newline
	Zwiebel in einem Topf in Olivenöl anbraten bis er glasig ist, danach Risotto dazu geben und mit
	einem Schuss Weißwein löschen. \newline Nun so viel Hühnerbrühe dazu bis der Reis bedeckt ist.
	Wenn die Brühe merklich verdunstet ist Gemüse dazu und wieder mit Brühe bedecken. \newline Falls Brokkoli
	verwendet wir diesen davor schon in Salzwasser Garen. \newline
	Brühe immer nach leeren wenn sie wieder stark verdunstet ist. Mit dem letzten Schöpfer Brühe auch
	den zerkleinerten Lachs samt allem Saft und Angebratenem in Pfanne dazu geben. \newline
	Wenn fast alle Brühe verdunstet ist eine gute Hand geriebenen Parmesan darunter mischen. \newline
	Mit Salz abschmecken und genießen.\newline\newline
	
	Risotto kann mit so gut wie allem Kombiniert werden. Gleich bleibt immer Olivenöl,Risotto Reis,
	Parmesan, Hühnerbrühe, Weißwein und Zwiebeln.\newline\newline
	
	Beispiele:
	\begin{itemize}
		\item Nur Käuter
		\item Nur Saffran (Beilage)
		\item Nur Curry (Beilage)
		\item Tomaten Risotto (Pizza Tomaten Sauce, Paprika und Oregano)
		\item Hühner Risotto (Statt Lachs einfach Hühnchen, gut mit Erbsen)
		\item Meeresfrüchte Risotto (Meeresfrüchte statt Lachs)
	\end{itemize}	
	
\end{addmargin}
