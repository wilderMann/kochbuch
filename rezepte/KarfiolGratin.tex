\chapter*{Karfiol Gratin}
\begin{multicols}{2}
 {\Large Zutaten}
 \begin{Zutaten}
		\item \sfrac{1}{2}kg Karfiol
		\item 1 Schuss Milch
		\item 1 Becher Obers/Creme Fina
		\item 1 Prise Pfeffer
		\item 1 Prise Salz
		\item 1 Prise Muskat
		\item 40g Parmesan <gerieben>
			
				
\end{Zutaten}
	
\columnbreak
\addPicture{karfiolGratin.jpg}
\end{multicols}

{\Large Zubereitung} \newline
\begin{addmargin}[1cm]{0cm}
	Karfiol in Salzwasser mit der Milch zusammen kochen bis Bissfest. (Schneller wenn man ihn zerteilt) \newline
	Während dessen Obers in einem Topf aufkochen und mit Salz, Pfeffer, Muskat würzen. \newline
	Ofen auf 220° vorheizen. \newline
	Karfiol aus dem Wasser nehmen und in vorn verteilen. Mit Obers übergießen und Parmesan darüber
	streuen. \newline
	15min im Rohr backen.\newline\newline
	
	Kann man auch gut Zucchini rein schneiden.\newline
	Sehr gut mit Reis.\newline
	Mittel gut mit Gemüse im Salzwasser.
	
\end{addmargin}
