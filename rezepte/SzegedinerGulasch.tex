\chapter*{Szegediner Gulasch}
\begin{multicols}{2}
 {\Large Zutaten}
 \begin{Zutaten}
		\item 50dag Gulaschfleisch
		oder 12dag Sojagschnetzeltes
		\item Schmalz(Schwein/Butter) oder Öl
		\item 30dag Zwiebeln
		\item 30-50dag Sauerkraut
		\item 1Tl Salz
		\item 1Tl Pfeffer
		\item 1El Kümmel
		\item 2Tl Paprikapulver
		\item 4 Knoblauchzehen
		\item 7dag Tomatenmark oder mehr
		\item 200ml Gemüsebrühe
		\item 1 mittelgroße Kartoffel
		\item 1Tl Sauerrahm pro Portion
		\item 150ml Naturjoghurt falls Fleischmager oder Sojageschnetzeltes
		
\end{Zutaten}
	
\columnbreak
\addPicture{szegedinerGulasch.jpg}
\end{multicols}

{\Large Zubereitung} \newline
\begin{addmargin}[1cm]{0cm}
	Fleisch oder Sojagschnetzeltes anbraten, ruhig bis es braun ist. \newline
	Zwiebeln dazu und mit anbraten. \newline
	Mit Mehl stauben, Salz, Kümmel, Paprika, Tomatenmark und Knoblauch dazu.
	Vermischen und bisschen anbraten, nicht zu viel weil Paprika bitter wird. \newline
	Mit der Gemüsebrühe aufgießen und Joghurt dazu. Gerne etwas mehr Wasser.
	Das ganze für 30min köcheln lassen. \newline
	Sauerkraut dazu, gut vermischen und weitere 30min köcheln lassen. Wasser dazu falls notwendig. \newline
	10min vor Ende die rohe Kartoffel reiben und dazu geben.
	Mit einem Tl Sauerrahm servieren. \newline
	Auch sehr sehr gut mit Kümmelkartoffeln.
	
	
\end{addmargin}
