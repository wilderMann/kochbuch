\chapter*{Gulasch}
\begin{multicols}{2}
 {\Large Zutaten}
 \begin{Zutaten}
		\item 50dag Zwiebeln
		\item 50dag Rindfleisch <zb. Schulter>
		\item Optional: Schwarte mit Fett
		\item Schweine Schmalz
		\item Salz
		\item Paprika Pulver
		\item 2 EL Tomatenmark
		\item 725ml heißes Wasser
		\item 125ml Rotwein
		\item Je nach Geschmack scharfes Paprika Pulver
		\item 2-3 Lorbeerblätter
		\item Etwas Kümmel
		\item Etwas Majoran
		\item Pfeffer
		\item 1 Suppenwürfel
		
		
		
\end{Zutaten}
	
\columnbreak
\addPicture{Gulasch.jpg}
\end{multicols}

{\Large Zubereitung} \newline
\begin{addmargin}[1cm]{0cm}
	Zwiebeln in halbieren und in Scheiben schneiden. Rindfleischleisch mit kaltem Wasser abspülen und
	mit Küchenpapier abtrocknen. Danach in ca. 3cm Stückchen schneiden.\newline
	Etwas Schmalz in einem Topf erhitzen, Schwarte im Ganzen und Rindfleisch dazu. Gut von allen Seiten
	anbraten, besonders die Schwarte.\newline
	Wieder etwas Schmalz dazu und den Zwiebel mit dem Fleisch Braten.\newline
	Zwiebel-Fleisch Gemisch mit Paprika <süß+ scharf>, Salz, Pfeffer und Majoran würzen. Tomatenmark
	untermengen.\newline
	Danach heißes Wasser und Rotwein dazu.\newline
	Etwas umrühren, Suppenwürfel hinein bröseln und 2-3 Lorbeerblätter dazu.\newline
	Gut durchmischen, zum Kochen bringen und dann auf mittlerer Hitze 11⁄4 bis 11⁄2 Stunden Garen lassen.\newline
	Nach einer Stunde Garzeit einmal umrühren und falls benötigt noch etwas Wasser zuführen.\newline
	Zum Schluss die Schwarte heraus nehmen und das Gulasch noch mit allen Gewürzen ab schmecken.\newline Fertig!\newline\newline
	
	Es können auch Kartoffelviertel mitgekocht werden, dazu ca. 30min vor Ende Kartoffeln schälen,
	vierteln und mitkochen lassen.\newline
	Schmeckt sehr gut mit Omas selbstgemachten Bandnudeln!
	
\end{addmargin}
