\chapter*{Asia Wok}
\begin{multicols}{2}
 {\Large Zutaten}
 \begin{Zutaten}
		\item Erdnuss Öl
		\item Hühner Fleisch/ Tofu/ Temphe
		\item Frühlings Zwiebeln
		\item Ingwer
		\item Rote Zwiebeln
		\item Bambus Sprossen
		\item Bohnen Keimlinge
		\item Karotten
		\item Champions
		\item Cashewkerne
		\item Ananas
		\item Soja Sprossen
		\item Erbsen Shote
		\item Soja Sauce
		\item 1Tl Maizena
		\item \sfrac{1}{2}Tl Suppen Würfel
		\item Wok Nudeln
		\item Wok Gewürz
		
		
					
		
\end{Zutaten}
\columnbreak
%\addPicture{tomatensuppe.jpg}
\end{multicols}

{\Large Zubereitung} \newline
\begin{addmargin}[1cm]{0cm}
	Wok erhitzen. Erdnussöl und klein geschnittenes Hühnchen Fleisch dazu. Leicht anbraten und wieder
	aus dem Wok heraus auf ein Teller.\newline
	Nun im Wok Zwiebeln glasig Braten, danach fein gehackter Ingwer dazu. Nach und nach das restliche
	Gemüse dazu je fester je früher.\newline Während das Gemüse mit geschlossenem Deckel etwas Gart ca. 1Tl
	Maizena und einen zerbröselten halben Tl Suppen Würfel in einen Joghurt Becher. Dazu etwas Soja
	Sauce und mit kaltem Wasser halb auffüllen, gut auflösen!\newline
	Danach Wok Nudeln mit Doppelt so viel Wasser zum Gemüse in den Wok geben und 5min geschlossen
	köcheln lassen.\newline Danach den Inhalt des Joghurt Bechers gut umgerührt dazu.\newline
	Nun gut Vermengen und mit Soja Sauce, Wokgewürz und Salz ab schmecken.\newline\newline
	
	Es kann Grundsätzlich so gut wie jede Zutat ausgewechselt werden. Erdnussöl ist nicht zwingend.\newline Es reicht wenn nur 2-3 Gemüsesorten verwendet werden, jedoch
	um so mehr um so gut.\newline
	Für den asiatischen Touch wichtig wären jedoch Soja Sauce, Wokgewürz, Ingwer und Knoblauch.
	
	
	
	
\end{addmargin}
