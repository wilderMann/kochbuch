\chapter*{Knoblauchcremesuppe}
\begin{multicols}{2}
 {\Large Zutaten}
 \begin{Zutaten}
		\item 1 Zwiebel
		\item 6-8 Zehen Knoblauch
		\item 20dag Kartoffeln
		\item 1EL Butter
		\item \sfac{1}{2}l Gemüsebrühe
		\item 100ml Schlagobers
		\item Salz
		\item Pfeffer
		\item 1Tl Kräuter
		\item Sauerrahm
		
		
				
		
\end{Zutaten}
\columnbreak
\addPicture{knoblauchcremesuppe.jpg}
\end{multicols}

{\Large Zubereitung} \newline
\begin{addmargin}[1cm]{0cm}
	Die Zwiebel und die Knoblauchzehen, je nach Geschmack können mehr oder weniger Zehen verwendet werden, abziehen und fein würfeln. Die Kartoffeln schälen und in kleine Stücke schneiden.\newline
	In einem Topf die Butter erwärmen und darin die Zwiebeln und den Knoblauch andünsten. Die Kartoffeln
	hinzufügen und mit Brühe aufgießen.\newline Die Suppe etwa 25 Minuten leise kochen lassen, kurz vor Ende der
	Garzeit die Sahne einrühren.\newline
	Die Suppe mit einem Mixstab cremig pürieren und mit Salz, Pfeffer und Küchenkräutern würzen. \newline In
	vorgewärmte Teller verteilen und mit je einem Klecks Sauerrahm garnieren.
	
	
	
	
\end{addmargin}
