\chapter*{Onigiri (Reisbällchen)}
\begin{multicols}{2}
 {\Large Zutaten}
 \begin{Zutaten}
		\item 1 Becher Sushi Reis (ca. 5 Onigiri)
		\item 2 Becher Wasser
		\item Geröstete Seegrasblätter
		\item Füllung
		
\end{Zutaten}
\columnbreak
%\addPicture{tomatensuppe.jpg}
\end{multicols}

{\Large Zubereitung} \newline
\begin{addmargin}[1cm]{0cm}
	Wasser leicht Salzen und in einem Topf zum kochen bringen. Reis dazu und geschlossen auf mittlerer
	Hitze 12 min köcheln lassen. Danach 15 min ohne Hitze geschlossen ausdampfen lassen.\newline
	Danach 1 Stück Küchenrolle ausbreiten (nicht zwingend) und gleich großes Stück Frischhaltefolie
	darauf legen.\newline Auf die Folie dann 2-3 Löffel Reis geben und ein bisschen ausbreiten. In die Mitte des
	ausgebreiteten Reis etwas von der Füllung geben.\newline
	Danach vorsichtig alle Lagen in beide Hände nehmen und eine Art Kelch bilden, in dessen Mitte sich
	die Füllung befindet. Nun versuchen einen Reisball zu formen ohne dass die Füllung nach aussen
	dringt.\newline Das ist reine Übungssache, mit festen Füllungen geht es am besten. Mit Preiselbeermarmelade
	fast unmöglich, von dieser am Besten mehr nehmen und sie Überfall gleichmäßig durchsickern lassen.\newline\newline
	Nun Versuchen aus dem Reisball ein gleichseitiges Dreieck zu formen und danach in Folie belassen.\newline
	Ein Blatt geröstetes Seegras in der Mitte teilen und der Breite nach 2-3cm Stücke herunter
	reißen. Das sollte ca. 8
	3cm Stücke ergeben.\newline Seegras am besten einige male falten und dann reißen.\newline
	Onigiri kann sowohl kalt als auch warm gegessen werden. Schmeckt lecker in Soja Sauce gedipt und mit
	Wasabi bestrichen.
	
	
	
	
	
\end{addmargin}
