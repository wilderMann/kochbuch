\chapter*{Krautwickler}
\begin{multicols}{2}
 {\Large Zutaten}
 \begin{Zutaten}
		\item 50dag-1kg Faschiertes
		\item \sfrac{1}{2}-1 Becher Reis
		\item 2-4 Eier
		\item 1-2 Packerl Sauerkraut <roh 1/2kg proPkg>
		\item 1 Zehe Knoblauch
		\item 1-2Tl Paprika Pulver
		\item Pfeffer
		\item Salz
		\item 2.5-5dag Schweineschmalz
		\item 2.5-5dag Mehl
		\item 1Tl Paprika Pulver
		\item Salz
		\item 1Tl Suppenpulver
		
		
		
		
		
		
\end{Zutaten}
	
\columnbreak
\addPicture{krautwickler.jpg}
\end{multicols}

{\Large Zubereitung} \newline
\begin{addmargin}[1cm]{0cm}
	Für 3 - 6 Personen.
	Faschiertes, Reis (roh), Eier, Knoblauch, Paprika, Pfeffer und Salz vermengen. Knödel formen, nicht
	zu groß, sie werden etwas aufgehen!\newline
	Sauerkraut abspülen.\newline
	In dem Topf den Boden mit Sauerkraut bedecken.
	Knödel mit genügend Platz lassen in Topf legen, Knödel in Sauerkraut Hüllen.\newline Knödel sollen sich
	selbst und die Wände nicht berühren.\newline
	Eventuell zweite Lage in Zwischenräumen.\newline
	Mit Wasser bedecken, ja nicht zu viel!\newline
	~1,5h köcheln lassen, dabei bei Bedarf wieder Wasser zugeben, aber auch wieder nicht zu viel!\newline\newline
	Für Einbrenn:\newline
	Schweineschmalz schmelzen und Mehl darin bräunen und verrühren. Gut braun, nicht zu dunkel.\newline
	Paprika, Suppenpulver und Salz einmischen.\newline
	Danach langsam mit kaltem Wasser vermengen bis Sauce cremig ist.\newline
	Zu den fertig gegarten Krautwickler die Einbrenn geben. Verrühren und fertig.\newline
	Servieren mit Brot. Kann gut mit Paprika gewürzt werden.\newline\newline
	Mama macht immer 1/2kg Faschiertes, 2Pkg Sauerkraut und 1/4 Becher Reis.
	(Reis könnte man nach oben tendieren, Sauerkraut ist etwas wenig wenn man viel Wasser nimmt)
	Ergibt ~8 Knödel
	
	
	
\end{addmargin}
