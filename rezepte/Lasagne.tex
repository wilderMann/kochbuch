\chapter*{Lasagne}
\begin{multicols}{2}
 {\Large Zutaten}\newline
 Rote Sauce:
 \begin{Zutaten}
		\item 25dag Faschiertes (\sfrac{1}{3}-\sfrac{1}{4} soviel Sojagranulat)
		\item 1 große Zwiebel
		\item 1pkg Passierte Tomaten
		\item Tomatenmark
		\item Weißwein
		\item Olivenöl
		\item 3 Knoblauchzehen
		\item Suppenwürfel (weniger mit Sojagranulat)
		\item Salz
		\item Pfeffer
		\item Rosmarin
		\item Oregano
		\item Basilikum
		\item Thymian
\end{Zutaten}
Bechamelsauce:
\begin{Zutaten}
	\item 3dag Mehl
	\item 3dag Butter
	\item \sfrac{1}{2}l Milch
	\item 3dag Parmesan	
\end{Zutaten}
	
\columnbreak
%\addPicture{shahiPaneer.jpg}
\end{multicols}

{\Large Zubereitung} \newline
\begin{addmargin}[1cm]{0cm}
	Rote Sauce wie Bolognese zubereiten.\newline
	Butter in Topf zergehen lassen, Mehl leicht bräunlich anbraten.
	Milch aufgießen und mit Schneebesen einrühren.\newline
	Unter ständigem Rühren aufkochen lassen, danach vom Herd ziehen und abkühlen lassen.\newline
	Parmesan untermischen.\newline
	Auflaufform passt perfekt der schwarze Bräter.\newline
	Reihenfolge ist Rot, Weiß, Teigblatte, Rot, Weiß....., Teigblatte, Rot, Weiß, Parmesan.\newline
	Man kann dann noch einige Butterflocken darüber tun.\newline
	\sfrac{1}{2} Stunde zugedeckt im Ofen bei 180°C. (Kann man auch mit Alufolie abdecken)
	Danach nochmal \sfrac{1}{4} Stunde offen.
	
	
\end{addmargin}
