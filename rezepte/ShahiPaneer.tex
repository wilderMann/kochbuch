\chapter*{Shahi Paneer}
\begin{multicols}{2}
 {\Large Zutaten}
 \begin{Zutaten}
		\item Paneer
		\item 2 kleine Zwiebeln
		\item 50dag passierte Tomaten
		\item 3-4 Zehen Knoblauch
		\item Ingwer
		\item Erbsen
		\item 4Tl Kreuzkümmel
		\item 4Tl Koriander
		\item 4Tl Garam Masala
		\item 1Tl Chilipulver
		\item \sfrac{3}{4} Becher Cashewnüsse
		\item 1 Becher Milch
		\item 1\sfrac{1}{2} Tassen Wasser
		\item 1El Zucker
		\item 1Tl Salz
		
		
		
		
\end{Zutaten}
\columnbreak
\addPicture{shahiPaneer.jpg}
\end{multicols}

{\Large Zubereitung} \newline
\begin{addmargin}[1cm]{0cm}
	Cashewnüsse in Milch einweichen.\newline
	Paneer in Öl/Butterschmalz goldgelb anbraten.\newline
	Zwiebel in Öl/Butterschmalz anbraten, Ingwer und Knoblauch dazu, kurz anbraten.
	Tomatensauce darüber, etwas köcheln.\newline
	Erbsen dazu.\newline
	Cashewnüsse pürieren.\newline
	Gewürze, Cashewmilch und Wasser dazu. Tendenziell bissl weniger Wasser.
	Mit Zucker und Salz abschmecken.\newline
	Zum Schluss Paneer dazu.
	
\end{addmargin}
