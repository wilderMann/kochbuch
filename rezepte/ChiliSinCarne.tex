\chapter*{Chili sin Carne}
\begin{multicols}{2}
 {\Large Zutaten}
 \begin{Zutaten}
		\item Zwiebel
		\item Reichlich Knoblauch gepresst
		\item Tomatenmark
		\item Pasata
		\item Gemüsebrühe
		\item Bohnen
		\item Linsen
		\item Kichererbsen
		\item Mais
		\item Tofu oder Sojagranulat
		\item Pilze
		\item Kartoffeln
		\item Pfeffer
		\item Chili
		\item Oregano
		\item Kreuzkümmel
		\item Zimt
		
\end{Zutaten}
	
\columnbreak
%\addPicture{shahiPaneer.jpg}
\end{multicols}

{\Large Zubereitung} \newline
\begin{addmargin}[1cm]{0cm}
	Das tolle an Chili, man kann so gut wie alles rein werfen.\newline
	Sojagranulat mit heißer Brühe und Soja Sauce quellen lassen. Falls man Tofu benutzt diesen anbraten und zur Seite geben. \newline
	Zuerst Zwiebeln anschwitzen, Knoblauch dazu. Gerne auch schon das Sojagranulat, wenn man lieber Tofu nimmt diesen erst später sonst zerkocht er sich.\newline
	Tomatenmark und Chilli darauf. Dann mit Gemüsebrühe und Pasata aufgießen.\newline
	Gemüße je nach Garzeiten zugeben und kochen lassen bis alles durch ist. \newline
	Mit den Gewürzen abschmecken.
	
	
\end{addmargin}
