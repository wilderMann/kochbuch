\chapter*{Apfelraderl}
\begin{multicols}{2}
 {\Large Zutaten}
 \begin{Zutaten}
		\item \sfrac{1}{2}l Joghurtbecher Mehl
		\item 1 Prise Salz
		\item \sfrac{1}{2}l Milch
		\item 4 Eier
		\item Äpfel
		
		
		
		
\end{Zutaten}
	
\columnbreak
%\addPicture{szegedinerGulasch.jpg}
\end{multicols}

{\Large Zubereitung} \newline
\begin{addmargin}[1cm]{0cm}
	Einen \sfrac{1}{2}l Joghurtbecher mit Mehl füllen, eine Prise Salz dazu und in einen Mixer geben.\newline
	Bevor man das Gerät einschaltet müssen die anderen Zutaten vorbereitet werden. Wenn durch das späte
	Zubereiten das zuführen der Zutaten verzögert wird bekommt man "Boban".\newline
	Also sollte man nun zwei Gefäße mit \sfrac{1}{4}l Milch, ein Gefäß mit den 4 aufgeschlagenen Eiern und
	optional ein Gefäß mit \sfrac{1}{8}l Bier <dann jedoch weniger Milch!> haben.
	Nun muss zuerst wärend der Mixer läuft \sfrac{1}{4}l Milch zugeführt werden.\newline Danach kommen gleich die Eier
	dazu. Alles gut durchmischen.
	Nun kurz eine Pause machen und mit einer Teigkarte die Ränder herunter putzen.\newline
	Danach die restlichen Zutaten dazu und nochmal gut Mixen.
	Alles gut in eine Schüssel putzen und nochmal die konsistenz prüfen.
	Teig darf nicht zu dünnflüssig sein.\newline
	Äpfel in Radln schneiden.
	Apfelradln in Teig ganz eintauchen und mit Löffel samt etwas Teig in die Pfanne geben.\newline
	Ähnlich wie Palatschinken herausbraten und mit Staubzucker servieren.\newline
	Falls zu wenig Teig Äpfel roh essen, falls zu wenig Äpfel dann Palatschinken machen.
	
	
	
\end{addmargin}
