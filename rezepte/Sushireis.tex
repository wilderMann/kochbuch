\chapter*{Sushireis}
\begin{multicols}{2}
 {\Large Zutaten}
 \begin{Zutaten}
		\item 1 Becher Sushi Reis
		\item 1,5 Becher Wasser
		\item 2EL Reisessig
		\item 1EL Zucker
		\item 1TL Salz
		
		
		
\end{Zutaten}
\columnbreak
\addPicture{sushireis.jpg}
\end{multicols}

{\Large Zubereitung} \newline
\begin{addmargin}[1cm]{0cm}
	Sushi-Reis in einem Sieb unter fließendem kaltem Wasser abspülen, bis das Wasser klar abläuft, und
	die Körner gut abtropfen lassen.\newline
	Den Reis mit Wasser aufkochen, 2 Minuten kochen, die Hitze reduzieren und den Reis
	zugedeckt bei geringer Hitze 10 Minuten ausquellen lassen.\newline
	Den Deckel abnehmen, 2 Lagen Küchenpapier zwischen Topf und Deckel klemmen und den Reis noch 10 bis
	15 Minuten abkühlen lassen.\newline
	In der Zwischenzeit Reisessig, Salz und Zucker aufkochen und wieder abkühlen lassen.\newline
	Den Reis in eine Schüssel füllen, den Würzessig darüber träufeln und mit einem Holzspatel
	unterarbeiten, dabei aber nicht rühren.\newline
	Den Reis bis zur weiteren Verwendung mit einem feuchten Tuch abdecken.\newline\newline
	
	3 Becher Reis reichen ca. für 10 Maki- Rollen.
	23dag Lachs für ca. 20 Sushi
	1 Becher Reis ergibt ca. 15 Sushi
	
	
	
\end{addmargin}
