\chapter*{Fruchtknödel}
\begin{multicols}{2}
 {\Large Zutaten}
 \begin{Zutaten}
		\item 1 Portion Topfen-Grieß Teig (ca. 10-12
		Knödel)
		\item 10-12 Marillen/Erdbeeren/Zwetschken
		\item Mehl
		\item Butter
		\item Zucker
		\item Semmelbrösel
		
		
\end{Zutaten}
	
\columnbreak
\addPicture{fruchtknödel.jpg}
\end{multicols}

{\Large Zubereitung} \newline
\begin{addmargin}[1cm]{0cm}
	Arbeitsfläche mit Mehl einreiben. Danach den fertigen Teig noch leicht durchkneten und eine Rolle
	mit ca. 4-5 cm Formen.\newline
	Diese Rolle dann in 10-12 gleich große Stücke schneiden. Hierbei ruhig öfter versuchen, die richtige
	Größe ist schwer.\newline
	Diese Teig Stücke danach auf der Handfläche flach drücken und das Obst damit umschließen und eine
	schöne Kugel rollen. Knödel danach noch kurz in Mehl rollen damit diese nicht auf der Arbeitsfläche
	ankleben.\newline
	In einer Pfanne Butter schmelzen, Brösel und Zucker darunter mischen.\newline
	Knödel während dessen in heißem Wasser ziehen lassen (nicht kochen!!) gleich nach dem ins Wasser
	gleiten lassen die Knödel vorsichtig umrühren damit sie nicht am Boden fest kleben.\newline
	Wenn die Knödel von selbst an die Oberfläche kommen sind sie fertig. Sie müssen aus dem Wasser
	genommen und in den Bröseln gewälzt werden.\newline Danach kann man sie servieren.\newline
	Gekochte Knödel können eingefroren und in Topf aufgetaut werden.\newline\newline
	
	Scharten Oma Variation: Marillen vorher halb aufschneiden, Kern entfernen und statt dessen
	Würfelzucker hinein.
	
	
\end{addmargin}
