\chapter*{Omas Nudelteig}
\begin{multicols}{2}
 {\Large Zutaten}
 \begin{Zutaten}
		\item 30dag griffiges Mehl
		\item 3 Eier
		\item 2-3 halbe Eischalen kaltes Wasser
		\item 1 Prise Salz
		
		
\end{Zutaten}
\columnbreak
\addPicture{grumbiranudla.jpg}
\end{multicols}

{\Large Zubereitung} \newline
\begin{addmargin}[1cm]{0cm}
	Etwas Wasser und die Eier versprudeln. Gemisch zum Mehl geb und kneten. Nach und nach das restliche
	Wasser dazu geben.\newline Der Teig wird so lange geknetet bis er fein und glatt ist.\newline
	Danach teilt man ihn in 3 bis 4 Stücke und walkt diese auf einem bemehlten Brett dünn aus.\newline
	Diese Flecken lässt man dann ca. 1h Übertrocknen und wendet sie dabei nach 30min.\newline
	Flecken dann in Handbreite streifen Schneiden und dann Nudelmaschiene zuführen. Falls keine
	Nudelmaschiene vorhanden ist kann man sie auch per Hand schneiden.\newline
	Mögliche daraus resultierende Nudelarten wären Suppennudeln, Bandnudeln oder Fleckerl.\newline Schmeckt auch
	gut als Teigplatten für Lasagne.\newline\newline
	Wenn sie nicht gleich verarbeitet werden die Nudeln gut trocknen lassen! \textbf{WICHTIG!}
	Ansonst können die Nudeln auch gleich Verarbeitet werden.
	\newline\newline
	Für Variationen Tomatenmark oder Spinat für rote oder grüne Nudeln verwenden.
	

\end{addmargin}
