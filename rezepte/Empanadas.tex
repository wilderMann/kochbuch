\chapter*{Empanadas}
\begin{multicols}{2}
 {\Large Zutaten}
 Teig:
 \begin{Zutaten}
		\item 2 Eier (1 für Teig/1 zum drüber streichen)
		\item 200ml Weißwein oder Wasser
		\item 41dag Mehl
		\item 1\sfrac{1}{2} TL Salz
		\item 1TL Backpulver
		\item 1EL Butter in feine Stückchen	
\end{Zutaten}
Füllung:
\begin{Zutaten}
		\item 5 Reife Tomaten oder 1 Dose Dosentomaten
		\item 1 Zwiebel mittelgroß
		\item 3 Zehen Knoblauch
		\item 1TL Salz
		\item \sfrac{1}{2} TL gemahlener Pfeffer
		\item \sfrac{1}{2} TL Zucker
		\item 5 gehäufte EL geriebener Käse
		\item 10 Blätter Basilikum
		
\end{Zutaten}
\columnbreak
\addPicture{empanadas.jpg}
\end{multicols}

{\Large Zubereitung} \newline
\begin{addmargin}[1cm]{0cm}
	1 Ei mit Flüssigkeit vermischen. Mehl mit Salz und Backpulver vermischen, dann mit der Butter
	zerbröseln.\newline
	\sfrac{3}{4} des festen in das flüssige mit einem Löffel verrühren.
	Rest des Mehls dazu und kneten bis es geschmeidig wird.
	20 Minuten rasten lassen.\newline
	Für die Füllung Tomaten kreuzweise einritzen, 45s in kochendes Wasser. Danach schälen und in Stücke
	schneiden. Zwiebel anschwitzen, Knoblauch kurz mit anbraten. Tomaten einrühren und bei mittlerer
	Hitze dünsten bis Flüssigkeit verdampft ist.\newline
	Salz, Pfeffer und Zucker zugeben.\newline\newline
	Füllung in eine Schüssel geben und mit Basilikum und Käse vermischen.
	Danach Teig dünn <2mm> ausrollen und ca. 12cm Kreise ausstechen.\newline
	In die Mitte der Kreise einen großen TL Füllung geben. Ränder mit Wasser benetzen und
	zusammenklappen.\newline
	Mit Gabel die Ränder nieder drucken.\newline
	Mit gequirrltem Ei bestreichen und ins Backrohr.\newline
	180°C Umluft für 20-25minb oder 200°C Ober- Unterhitze.
	
\end{addmargin}
