\chapter*{Gefüllte Paprika}
\begin{multicols}{2}
 {\Large Zutaten}
 \begin{Zutaten}
		\item 5 Paprika
		\item \sfrac{1}{2}kg Faschiertes
		\item \sfrac{1}{2} Tasse Reis
		\item 2 Eier
		\item 1 P. Tomatenmark
		\item Salz
		\item Paprika Pulver
		\item 1 Löffel Schweineschmalz
		\item 2 Löffel Mehl
		\item 1 Löffel Zucker
		
		
		
		
		
\end{Zutaten}
	
\columnbreak
%\addPicture{szegedinerGulasch.jpg}
\end{multicols}

{\Large Zubereitung} \newline
\begin{addmargin}[1cm]{0cm}
	Paprika waschen und aushöhlen. \sfrac{1}{2}kg Faschiertes, \sfrac{1}{2} Tasse Reis, 2 Eier, Salz und Paprika Pulver
	gut vermischen. \newline
	Die Paprika füllen und die restliche Masse zu Kugeln formen. \newline
	Gefüllte Paprika, Kugeln und \sfrac{1}{2} P. Tomatenmark in einen Topf geben. Danach Wasser dazu bis alles
	bedeckt ist. Zum Kochen bringen. \newline
	Danach in einem extra Topf eine Einbrenn anrichten. Dazu 1El Schweineschmalz, 2 El Mehl und 1 El
	Zucker erhitzen, es soll jedoch nicht zu dunkel werden. \newline Dazu dann vorsichtig (es spritzt) die zweite
	hälfte Tomatenmark und etwas Wasser geben und zum Kochen bringen. \newline Danach über die gefüllten Parika
	gießen und dazu nach gutdüngen noch Wasser dazu. \newline Noch einmal zum Kochen bringen.
	
	
	
	
\end{addmargin}
