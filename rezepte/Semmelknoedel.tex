\chapter*{Semmelknödel}
\begin{multicols}{2}
 {\Large Zutaten}
 \begin{Zutaten}
		\item 1 mittelgroße Zwiebel
		\item 25dag Semmelwürfel
		\item 4 Eier
		\item Etwa \sfrac{1}{4}l Milch \left(übrig gebliebener Schlagobers oder Sauerrahm geht auch\right) 
		\item Petersilie
		\item Pfeffer
		\item Salz
		
		
\end{Zutaten}
\columnbreak
\addPicture{semmelknoedel.jpg}
\end{multicols}

{\Large Zubereitung} \newline
\begin{addmargin}[1cm]{0cm}
	Für ungefähr 6-8 Knödel.\newline\newline
	
	Zwiebel würfelig schneiden und in etwas Butter anschwitzen.\newline
	Platte ausschalten und Milch hinzugeben.\newline
	Eier dazu schlagen und mit Schneebesen verquirlen.\newline
	Pfeffer, Salz, Gewürze und Petersilie dazu.\newline
	Semmelwürfel in eine große Schüssel geben.
	Gemisch gleichmäßig darüber leern und dabei umrühren.\newline
	Es sollte genug Flüssigkeit sein damit sich das Knödelbrot voll saugen kann, bei Bedarf etwas Milch
	dazu geben. \left(Es sollte etwas Flüssigkeit am Boden der Schüssel stehen\right) \newline
	Locker 30min stehen lassen. Die einzelnen Semmelwürfel sollten sich leicht zwischen den Fingern
	zerdrücken lassen. Während dem warten gelegentlich umrühren.\newline
	Hände befeuchten und Knödel formen. Diese dann auf einem Dämpfeinsatz mit Küchenpapier legen.\newline
	Funktioniert auch toll als Semmelrolle, dafür in Backpapier oder ein Geschirrtuch einrollen.\newline
	Wasser einfüllen so das der Dampfeinsatz nicht überschwemmt wird.\newline Etwa 20min dämpfen.
	
	
\end{addmargin}
